\documentclass{article}
\begin{document}

\title{Box Membrane - Plane Wave Project}
\author{}
\date{\today}
\maketitle

\section{Introduction}
This document will track all steps in the development of the Box Membrane - Plane Wave project. The project consists of two main phases:
\begin{itemize}
    \item \textbf{Phase 1:} Setting up the numerical solutions for the wave equation on a stretched membrane.
    \item \textbf{Phase 2:} Using the numerical results to animate the interaction in Blender.
\end{itemize}

\section{Project Setup}
The project directory was created, and version control was initialized with Git. The project was then linked to a GitHub repository for remote access.

\section{Numerical Solution Setup}

We begin by defining the necessary parameters for simulating the wave equation on a rectangular membrane.

\subsection{Simulation Parameters}
The membrane dimensions are set to 1m by 1m, and the wave speed is chosen to be 1 m/s. The grid is discretized into 100 points along each dimension, resulting in a grid spacing of $\Delta x = \Delta y = 0.01$ m. We use a time step of $\Delta t = 0.001$ s for stability. 

\subsection{Initialization of Displacement Arrays}
Three displacement arrays are initialized:
\begin{itemize}
    \item $u$: Displacement at the current time step.
    \item $u_{\text{prev}}$: Displacement at the previous time step.
    \item $u_{\text{next}}$: Displacement at the next time step.
\end{itemize}

The initial conditions are set so that the membrane is at rest, with zero initial displacement and velocity.


\end{document}
